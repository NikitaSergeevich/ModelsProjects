\documentclass{article}
\usepackage{zed-csp}
%\input{handout}

\begin{document}
	
	There are several state variables that will describe state of the pump:\\
	ALARMSTATE: Is alarm activated, silenced or turned off\\
	PARAMSSTATE: Are parameters set or not set\\
	UNITLOCKSTATE: Is unit locked or unlocked \\
	LINESLOCKSTATE: Is line locked, unlocked, hooked up or hooked out\\
	PUMPSTATUS: Current status of the pump (pump or not pump)\\
	PUMPERROR: Status of the possible pump error during the lifetime\\
	LINESTATUS: Status of the line events
	POWERMODE: Possible states of the power
	
	\begin{zed}	
		
		
		ALARMSTATE ::= activated | silenced | alarm\_off \\
		
		PARAMSSTATE ::= set | not\_set \\
	
		UNITLOCKSTATE ::= unit\_locked | unit\_unlocked \\
	
		PUMPSTATUS ::= pump\_on | pump\_off \\		
		
		PUMPERROR ::= ok | end\_of\_treatment\_time | post\_error | system\_check\_error \\ | run\_out\_medicine | low\_battery | line\_issue \\
		
		LINEERROR ::= ok | med\_ended | pinched | plugged \\
		
		LINESTATUS :: = line\_locked | line\_unlocked | line\_hooked\_out \\
		
		POWERMODE ::= electricity | battery | low\_battery | empty\_battery \\
	\end{zed}
	
	//Problem - we can have only one state
	
	\begin{schema}{Line}
		fluid: \nat \\
		line\_status: LINESTATUS \\
		line\_error: LINEERROR 
	\end{schema}
	
	\begin{zed}
		ID::=1|2|3
	\end{zed}
		
	\begin{schema}{Pump}
		lines : ID \pfun Line \\
		alarmstate : ALARMSTATE \\
		unitstate : UNITLOCKSTATE \\
		powermode : POWERMODE \\
		status: PUMPSTATUS \\
		paramset:PARAMSSTATE
	\where
		\# \ran lines = \# \dom lines\\
	\end{schema}
	
	At the beginning pump has no lines plugged in and powered of.
	Unit and lines state is unlock. Alarm is not sound. 
	Pump has three inputs.	
	
	\begin{schema}{InitPump}
		Pump
		\where
		lines = \emptyset \\
		alarmstate = off \\
		unitstate  = unlocked \\
		powermode =  no\_power \\
		status = pump\_off\\
		param\_set = not\_set	
	\end{schema}
	
	\begin{schema}{InitLine}
		Line
		\where
		fluid = 500 \\
		line\_status = line\_hooked\_out \\
		line\_error = ok \\
	\end{schema}
	
	Dataframe
	\begin{schema}{PumpOp}
		\Delta Pump
	\where
    	activeids' = activeids \\
    	lines' = lines  \\
    	alarmstate' = alarmstate \\
    	linelockstate' = linelockstate  \\
    	unitstate' = unitstate \\
    	powermode' = powermode \\
    	status' = status \\
    \end{schema}
	
	Level of abstraction set 
	\begin{schema}{SetParam}
		PumpOp \\
		paramset? : PARAMSSTATE 
	\where
		status = pump\_on \\ 
		unitstate = unit\_unlocked \\
		paramset' = paramset?
	\end{schema}
	
    \begin{schema}{UnitLockUnlock}
		\Delta Pump \\
		new\_unitstate? : LINELOCKSTATE
	\where
		status = pump\_on \\ 
		unitstate \neq new\_unitstate? \\
		unitstate' = new\_unitstate? \\
    	lines' = lines  \\
    	alarmstate' = alarmstate \\
    	powermode' = powermode \\
    	status' = status \\	
	\end{schema}
	
	When we adding a new line, the id parameter identifies number of 
	one of the three input inside of the infusion pump.	
	
    \begin{schema}{LineLock}
		\Delta Pump \\
		id1? : ID \\
		line? : Line \\
	\where
	    pump shouldn't work? \\
		id1? \notin \dom lines \\
    	lines' = lines \cup \{id1? \mapsto line?\} \\
    	alarmstate' = alarmstate \\
    	unitstate' = unitstate \\
    	powermode' = powermode \\
    	status' = status \\
    	paramset' = paramset
	\end{schema} 	

	
	When we removing a line from the pump, it should have unlock state
	and also have id parameter which identifies number of one of the
	three input of the infusion pump.

	\begin{schema}{LineUnlock}
		\Delta Pump \\
		id1? : ID \\
	\where
		pump shouldn't work? \\
		id1? \in \dom lines \\
		lines' = lines \setminus \{id1? \mapsto lines(id1?)\} \\
    	alarmstate' = alarmstate \\
    	unitstate' = unitstate \\
    	powermode' = powermode \\
    	status' = status \\
    	paramset' = paramset
	\end{schema}

	
	\begin{schema}{RefillLineMedicine}
		\Delta Line \\
		amount?: \nat \\
	\where
	    fluid + amount? \leqslant 500 \\
	    fluid' = fluid + amount?	
	\end{schema}
	
	To start working with the pump we removing a line from the pump, it should have unlock state
	and also have id parameter which identifies number of one of the
	three input of the infusion pump.	
	Covers when we lost electricity
	\begin{schema}{PlugIn}
		\Delta Pump\\
		status? : PUMPSTATUS
	\where
		status \neq status? \\
		status' = status? \\
		alarmstate' = alarmstate \\
		unitstate' = unitstate \\
		status? = plug\_out \implies powermode' = battery \\
		status? = plug\_in \implies powermode' = electricity \\
		status' = status \\
		paramset' = paramset		
	\end{schema}
	
	\begin{schema}{SwitchPowerMode}
		\Delta Pump \\
		mode? : POWERMODE
	\where 
		status = plug\_in \\
		status' = status \\
		alarmstate' = alarmstate \\
		unitstate' = unitstate \\
		status? = plug\_out \land mode? = electricity \implies powermode' = battery \\
		status? = plug\_out \land mode? = battery \implies powermode' = battery \\
		
		status? = plug\_in \implies powermode' = mode? \\
		mode? = battery \\
		mode = elictricity \\
		
		powermode' = mode? \\
		status' = status \\
		alarmstate' = alarmstate \\
		linelockstate' = linelockstate \\
		unitstate' = unitstate \\
	\end{schema}
	
	%Power On Self Test
	\begin{schema}{TurnOn}
		\Delta Pump \\
		status?:PUMPSTATUS
	\where 
		status? = post\_error  \\
		status? = plug\_on \\
		status' = status?
	\end{schema}
			
	\begin{schema}{SystemCheck}
		\Delta Pump\\
		status?:PUMPSTATUS
	\where 
		status? = system\_check\_error \lor system\_check\_pass \\
		status? \neq status\\
		status' = status?
	\end{schema}
	
	Pushing fluids through the particular pump line (add frame) alarm is a pump status, how to obtain it from
	
	\begin{schema}{PushFluidsPromote}
		\Delta Pump	\\
		\Delta Line \\
		id1? : ID \\
	\where 
		id1? \in activeids \\
		(lines id1?) = \theta Line \\
		lines' = lines \oplus \{id1? -> \theta Line'\} \\		
	\end{schema}
	
	//Add alarm here + runoutofmedicene
	\begin{schema}{DispenseFluids}
		\Delta Line \\
		amount?: \nat \\
		status?: LINESTATUS
	\where 
		fluid > what? \\
		fluid' = fluid - amount? \\
		line_status' = LINESTATUS
	\end{schema}

\end{document}
