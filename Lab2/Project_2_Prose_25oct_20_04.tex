\documentclass{article}
\usepackage{times}
\usepackage{fullpage}

\newcommand{\head}{\subsection*}
\setlength{\parindent}{0pt}

\begin{document}

%_______________________________________________________________________________
% Heading section
%_______________________________________________________________________________
\begin{center}
\rule{\textwidth}{1.5pt} \\ \rule[10pt]{\textwidth}{1pt}\\
MSIT-SE-M-01\hfill Models of Software Systems\\[3ex]
{\Large\bf Project 2: Z Modeling}\\[3ex]
Innopolis University \hfill {\bf Due: October 28, 2015} \rule{\textwidth}{1pt}
\\\rule[9.5pt]{\textwidth}{1.5pt}
\end{center}

The purpose of this project is to give you experience in modeling a
realistic system as a Z specification. The example that we will use
is the Infusion Pump. A general description of an Infusion Pump can
be found in the General Project Documents folder on the class
blackboard.

\bigskip You should carry out this project in your assigned team.
Make sure that everyone in the group contributes to the overall
effort. Each team should submit a single write-up of the project,
due at the beginning of class on the project due date. We have
posted a template for a group project write-up under the
\LaTeX~section of the course web site.

%___________________________________________________________________________________________________
\head{Task 1 (90 points):}
%___________________________________________________________________________________________________

From a certain level of abstraction, a machine with multiple,
concurrent threads may be considered to be a machine with a single
thread, particularly if we do not permit events to occur
simultaneously. From this point of view, it is possible to treat an
N-channel infusion pump as a sequential system where events may
occur in any one channel, but at different times. Moreover, the
specification of events' timing is a non-trivial task and we can
abstract away when the events must occur.

\bigskip Your first task is to develop a specification of a
3-channel pump. Use the description of the model from the Blackboard
site to define the requirements for the pump. Your model should show
how the pump operates  in the presence of failures and, also,
consider the behavior in the presence or absence of electric power.
As a hint, consider separating the model of a channel from the model
of a pump that houses 3-channels. You should use promotion (just as
in the Library Problem) to extend a model of a channel to the model
of the pump.

\bigskip Be sure to document your specification with appropriate explanatory prose

\bigskip Effective use of the schema calculus will be one of the evaluation criteria for your specification.

\bigskip Be sure to typecheck your specification using fuzz,
Z-Eves, or CZT.

\bigskip Answer the following questions in your project write-up:
\begin{enumerate}
    \item Describe any decisions that your team made in resolving
    ambiguities in the English description. Also describe
    alternatives you considered and rejected.
    \\
    \textbf{Answer:}\\
    
   There is an ambiguity about the fact if a patient can modify the pump settings. For this, we make an assumption that patient can't make Unlock unit action because it is protected by some mechanism (only caregiver or doctor can use it).
    
    Pump specification has ambiguity of the pump behaviour in case of electricity problems. In our specification in case of electricity loss, pump will automatically switch to the battery.
    
    Silenced alarm is not zero volume but quite volume.
    According to requirements from ”Generic Infusion Pump Hazard Analysis and Safety Requirements”, we include assumption that pump will work on low battery level less than 4 hours.
    
	There is no description how the power-on-self-test works, so we've added the checked options into the specification.
 
    \item Would your model be more complicated if there were more
    channels?
    \\
    \textbf{Answer}\\
    No. We should only modify enumeration $ID$ by adding new channel numbers and change several places where the cardinality of $lines$ set is used.
    \item What happens when the pump is out of liquid?
    \\
    \textbf{Assumption:}\\
    If the pump is out of liquid it means that some (or one) of pump's lines are out of liquid.
    \textbf{Answer:}\\
    In this case alarm will be set on if the $dispense\_fluids$ action will be made for the particular line 
    
    \item Is it possible to dispense medicine without first
    setting a rate?
    \\
    \textbf{Answer:,,,,,,}\\
    Nikita will do it.
    \item Is it possible to start the pump when the key is locked?
    \\
    \textbf{Answer: YES}\\
    We can lock or unlock the line by special operation $Linelockorunlock$.. What is line lock and unit lock? What is key? 
    \item What happens when the the pump is out of power?
    \\
    \textbf{Answer:ARTUR}\\
    Pump would go the state PUMPSTATUS $pump\_off$ and the $POWERSTATUS$ would stay in previous condition...Describe in details when battery empty?? 
    \item Are there any extensions in your model not defined in
    the written description of the pump?
    \\
    \textbf{Answer:}\\
    Wait for the answer what is description? *Battery
\end{enumerate}

\noindent The full specification should be attached to your project
write-up. Hand in a hard copy and email a soft copy to the course
teaching assistants before class on the due date.

%___________________________________________________________________________________________________
\head{Task 2 (10 Points):}
%___________________________________________________________________________________________________

For the second part of your project you should write a short essay
(less than 1 page) that speculates on a more-complete Z model of the infusion pump. In particular, consider how you might use the Z schema calculus to introduce start- and end-times and duration into your model.
Additionally, reflect on the experience of developing a model in a
group; concentrate on how the formal model helped or hindered you in
understanding the infusion pump.


\end{document}
