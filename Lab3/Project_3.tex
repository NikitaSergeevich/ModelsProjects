\documentclass{article}
\usepackage{times}
\usepackage{fullpage}

\newcommand{\head}{\subsection*}
\setlength{\parindent}{0pt}

\begin{document}

%_______________________________________________________________________________
% Heading section
%_______________________________________________________________________________
\begin{center}
\rule{\textwidth}{1.5pt} \\ \rule[10pt]{\textwidth}{1pt}\\
MSIT-SE-M-01\hfill Models of Software Systems\\[3ex]
{\Large\bf Project 3: Concurrency}\\[3ex]
Innopolis University \hfill {\bf Due: November 18, 2015} \rule{\textwidth}{1pt}
\\\rule[9.5pt]{\textwidth}{1.5pt}
\end{center}

The purpose of this third project is to give you experience in
modeling a realistic system as a state machine using concurrency.
The example that we will use is the (by now) familiar Infusion Pump.
A general description of an Infusion Pump can be found in the
General Project Documents folder on the class blackboard. The key
ideas that we would like you to get out of this project are: (1) the
use of concurrency to manage complexity, separate concerns, model
reality; (2) checking of properties related to concurrency (safety
and liveness); and (3) additional practice in creating appropriately
abstract models.

\bigskip You should carry out this project in your assigned team. Make sure that everyone in the
group contributes to the overall effort. Each team should submit a single write-up of the project,
due at the beginning of class on the project due date. We have posted a template for a group
project write-up under the Latex section of the course web site.

%___________________________________________________________________________________________________
\head{Task 1 (50 points): Modeling with Concurrency}
%___________________________________________________________________________________________________

Model a 2-line infusion pump in FSP, using concurrency to factor the model into parts that
represent different concerns. Possibilities for separation include things like (a) power system (b) an individual line
(c) alarms (d) user interface for setting of the pump (both initially and during operation).

\bigskip

As always, you will need to pick a level of abstraction appropriate for this model, and it is up to
you to figure out what are the significant aspects of the system that should be included in your
model.

%___________________________________________________________________________________________________
\head{Task 2 (32 points): Stating and Checking Properties}
%___________________________________________________________________________________________________

Once you have your pump specified, consider the following properties. For each say (a) whether the property
is a safety or liveness property,  (b) whether your model allows you to check this property, and if
so, (c) whether it is true or not, and what features of FSP and LTSA allowed you to check the
property. It would be particularly helpful if you include in your write-up the specific checks that you performed -- which should also appear in your FSP file. (Note: not all of these properties have to be true of your pump, depending on how you interpret the requirements for an infusion pump.)

\begin{enumerate}
    \item The pump cannot start pumping without the operator first confirming the settings on the pump.
    
    \textbf{Answer:}\\
    It is a safety property, because it asserts that nothing bad happens during execution. In this case, that it is impossible to start the pump without confirming the settings. As it was said in our model description, modifying settings happens before the \verb|dispense| action. To check this we added property to LTSA scheme:\\
    \textbf{add here property}
    
    \item Electrical power can fail at any time.
    \textbf{Answer:}\\
    It is a liveness property, because it describes that something bad will eventually happen. Our model can allows to check this property, to be more precise, we can show, that electrical filure (\verb|plugOff| action) can appear infinitely often:
     \textbf{add here property}
      
    
    \item If the backup battery power fails, pumping will not occur on any line.
    \item It is always possible to resume pumping after a failure.
    \item An alarm will sound on any line failure (blockage, pinching, empty fluid, or
whatever failures you model).
    \item In the absence of errors the pump will continue to pump until the
treatment is finished.
    \item The system never deadlocks.
    \item Two other properties of your choosing.
\end{enumerate}

Now come up with two more properties that are not in the above list, and check them.
%___________________________________________________________________________________________________
\head{Task 3 (18 points): Reflection}
%___________________________________________________________________________________________________

 You have now seen three notations for specifying systems and their properties: Pre-post conditions,
 Z (Z/Eves and Fuzz), and FSP (LTSA).  In this part of the report we would like you to reflect on that experience.
 For each of these notations write a paragraph or two explaining:
\begin{enumerate}
 \item What are the strengths of this notation and its tools?  Under what situations would you
recommend its use? Why?
 \item What are the weaknesses of this notation and its tools. Under what
situations would you not not recommend its use? Why?
 \item With respect to this notation, what is
the single most-important future development that would be needed to make it more generally useful
to practitioners?

\end{enumerate}

\end{document}
