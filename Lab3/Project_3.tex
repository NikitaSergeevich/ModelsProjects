\documentclass{article}
\usepackage{times}
\usepackage{fullpage}

\newcommand{\head}{\subsection*}
\setlength{\parindent}{0pt}

\begin{document}

%_______________________________________________________________________________
% Heading section
%_______________________________________________________________________________
\begin{center}
\rule{\textwidth}{1.5pt} \\ \rule[10pt]{\textwidth}{1pt}\\
MSIT-SE-M-01\hfill Models of Software Systems\\[3ex]
{\Large\bf Project 3: Concurrency}\\[3ex]
Innopolis University \hfill {\bf Due: November 18, 2015} \rule{\textwidth}{1pt}
\\\rule[9.5pt]{\textwidth}{1.5pt}
\end{center}

The purpose of this third project is to give you experience in
modeling a realistic system as a state machine using concurrency.
The example that we will use is the (by now) familiar Infusion Pump.
A general description of an Infusion Pump can be found in the
General Project Documents folder on the class blackboard. The key
ideas that we would like you to get out of this project are: (1) the
use of concurrency to manage complexity, separate concerns, model
reality; (2) checking of properties related to concurrency (safety
and liveness); and (3) additional practice in creating appropriately
abstract models.

\bigskip You should carry out this project in your assigned team. Make sure that everyone in the
group contributes to the overall effort. Each team should submit a single write-up of the project,
due at the beginning of class on the project due date. We have posted a template for a group
project write-up under the Latex section of the course web site.

%___________________________________________________________________________________________________
\head{Task 1 (50 points): Modeling with Concurrency}
%___________________________________________________________________________________________________

Model a 2-line infusion pump in FSP, using concurrency to factor the model into parts that
represent different concerns. Possibilities for separation include things like (a) power system (b) an individual line
(c) alarms (d) user interface for setting of the pump (both initially and during operation).

\bigskip

As always, you will need to pick a level of abstraction appropriate for this model, and it is up to
you to figure out what are the significant aspects of the system that should be included in your
model.

%___________________________________________________________________________________________________
\head{Task 2 (32 points): Stating and Checking Properties}
%___________________________________________________________________________________________________

Once you have your pump specified, consider the following properties. For each say (a) whether the property
is a safety or liveness property,  (b) whether your model allows you to check this property, and if
so, (c) whether it is true or not, and what features of FSP and LTSA allowed you to check the
property. It would be particularly helpful if you include in your write-up the specific checks that you performed -- which should also appear in your FSP file. (Note: not all of these properties have to be true of your pump, depending on how you interpret the requirements for an infusion pump.)

\begin{enumerate}
    \item The pump cannot start pumping without the operator first confirming the settings on the pump.
    
    \textbf{Answer:}
    \begin{enumerate}
    \item It is a safety property, because it asserts that nothing bad happens during execution. In this case, that it is impossible to start the pump without confirming the settings.
    \item  Our model allows to check the property. As it was mentioned in our model description, we have modelled only one type of settings - setting a rate for a concrete infusion line. Therefore, \verb|dispense| action happens only after \verb|setRate[i]| action.
    \item Yes, it is true. Here is the property and LTSA output.
	\begin{verbatim}
fluent SETTINGS_ARE_SET[i:RangeLine] =
<line[i].setRate[j:MinRate..MaxRate], {turnOff, plugOffToEmptyBattery}>
assert DISPENSE_ONLY_AFTER_SETTINGS =
forall[i:RangeLine] [](line[i].dispense -> SETTINGS_ARE_SET[i])

_______________________________________________
...
Depth 27 -- States: 156103 Transitions: 1379653 Memory used: 292890K
No deadlocks/errors
Analysed in: 473ms
	\end{verbatim}    
    \end{enumerate}
    
    \item Electrical power can fail at any time.\\
    \textbf{Answer:}
    \begin{enumerate}
    	\item It is a liveness property, because it describes that something bad will eventually happen.
    	\item Our model allows to check this property. As it was mentioned in our model description, our pump model doesn't make differense between cases when pump was plugged off from AC or an electrical failure happened.
    	\item Yes, it is true. We can show, that electrical failure (\verb|plugOff| action) can appear infinitely often.
    	
    	\begin{verbatim}
    	assert ELECTRICAL_FAILURE_AT_ANY_TIME = []<> plugOff
    	_________________________________________
    	-- States: 390726 Transitions: 4734710 Memory used: 200917K
    	No LTL Property violations detected.
    	LTL Property Check in: 2064ms
    	\end{verbatim}
    \end{enumerate}
     
     
      
    
    \item If the backup battery power fails, pumping will not occur on any line.\\
    \textbf{Answer:}
    \begin{enumerate}
    \item It is safety property. According to our specification pump can dispense fluids if it works from power
    socket (electricity) or battery in case if battery is not empty. Another words, nothing wrong (pumping on any line) will
    not occur in case backup battery power fails.
 
    \item Our model allows to check this property. According to our assumption, backup battery fail is represented by fluent
    $DISCHARGED$ and can occur in two cases:\\
    1. If pump works from electricity with empty battery and electricity fails by \textbf{Check it:} $plugOffToEmptyBattery$
    $plug\_off$ action.\\
    2. If pump works from battery and battery become empty during pumping (by double $discharge$ action).
    We can fix this issue and continue dispense by using $chargeBattery$ or $plugIn$ action.\\
    \textbf{Check it:} Assert or Property NODISPENSE tells us that $dispense$ action is never available during battery fail.
    \begin{verbatim}
    fluent DISCHARGED = <plugOffToEmptyBattery, {chargeBattery, plugIn}>
    assert NODISPENSE = [](DISCHARGED -> !line[i:RangeLine].dispense)
    \end{verbatim}
    
    \item NODISPENSE property is not violated in our model. LTSA output.
    \begin{verbatim}
    Violated.
    \end{verbatim} 
    \end{enumerate}    
    
    
    \item It is always possible to resume pumping after a failure.
    \item An alarm will sound on any line failure (blockage, pinching, empty fluid, or
whatever failures you model).\\
    \textbf{Answer:}
\\
a) safety property for empty fluid - in this case, will be alarm\\
b) during the LINE process\\
c) we check this using "when" condition inside of LINE process (by comparing values of volume and rate)


    \item In the absence of errors the pump will continue to pump until the
treatment is finished.
\\
    \textbf{Answer:}\\
a) ? (using previous)\\
b) (only for case with empty fluid)\\
c) (similarly)
    \item The system never deadlocks.\\
    \textbf{Answer:}
    \begin{verbatim}
    Composition:
EVERYTHING = PUMP || line.1:LINE || line.2:LINE || line.1:ALARM || line.2:ALARM || BATTERY || line.1:SAFE_DISPENSE || line.2:SAFE_DISPENSE
State Space:
 4 * 102 * 102 * 6 * 6 * 3 * 2 * 2 = 2 ** 26
Composing...
Depth 37344 -- States: 10000 Transitions: 91019 Memory used: 69586K
Depth 58240 -- States: 20000 Transitions: 174654 Memory used: 84214K
Depth 64991 -- States: 30000 Transitions: 256613 Memory used: 98240K
Depth 68622 -- States: 40000 Transitions: 339361 Memory used: 112318K
Depth 58253 -- States: 50000 Transitions: 423537 Memory used: 88841K
Depth 46493 -- States: 60000 Transitions: 506861 Memory used: 102857K
Depth 34069 -- States: 70000 Transitions: 589260 Memory used: 116380K
Depth 21875 -- States: 80000 Transitions: 674665 Memory used: 130385K
Depth 9779 -- States: 90000 Transitions: 769373 Memory used: 106446K
-- States: 97998 Transitions: 843738 Memory used: 152722K
Composed in 974ms
    \end{verbatim}
    As we can see system is not in a deadlock anytime.
    \item Two other properties of your choosing.
\end{enumerate}

Now come up with two more properties that are not in the above list, and check them.
%___________________________________________________________________________________________________
\head{Task 3 (18 points): Reflection}
%___________________________________________________________________________________________________

 You have now seen three notations for specifying systems and their properties: Pre-post conditions,
 Z (Z/Eves and Fuzz), and FSP (LTSA).  In this part of the report we would like you to reflect on that experience.
 For each of these notations write a paragraph or two explaining:
\begin{enumerate}
 \item What are the strengths of this notation and its tools?  Under what situations would you
recommend its use? Why?
 \item What are the weaknesses of this notation and its tools. Under what
situations would you not not recommend its use? Why?
 \item With respect to this notation, what is
the single most-important future development that would be needed to make it more generally useful
to practitioners?

\end{enumerate}

\end{document}
